\chapter{Теоретические вопросы}

\section{Элементы языка: определение, синтаксис, представление в памяти}

Вся информация (данные и программы) в Lisp представляется в виде символьных выражений~--- S~-~выражений.

По определению \texttt{S~-~выражение ::= <атом>|<точечная пара>}

Атомы бывают:
\begin{itemize}
	\item символы (идентификаторы)~--- синтаксически~--- набор литер (букв и цифр), начинающихся с буквы;
	\item специальные символы~--- {Т, Nil} (используются для обозначения логических констант);
	\item самоопределимые атомы~--- натуральные числа, дробные числа (например 2/3), вещественные числа, строки~--- последовательность символов, заключенных в двойные апострофы (например <<abc>>).
\end{itemize}

Более сложные данные~-- списки и точечные пары (структуры) строятся из унифицированных структур~-- блоков памяти~-- бинарных узлов.

\texttt{Точечные пары ::= (<атом>.<атом>) | (<атом>.<точечная пара>) |(<точечная пара>.<атом>) | (<точечная пара>.<точечная пара>);}

\texttt{Список ::= <пустой список> | <непустой список>, где
<пустой список> ::= ( ) | Nil,
<непустой список> ::= (<первый элемент>.<хвост>),
<первый элемент> ::= <S-выражение>, <хвост> ::= <список>.}

Синтаксически: любая структура (точечная пара или список) заключается в круглые скобки ( A . B )~--- точечная пара, ( А ) ~--- список из одного элемента, пустой список изображается как Nil или ( ).

\clearpage

\section{Особенности языка Lisp. Структура программы. Символ апостроф}

Особенности языка Lisp:
\begin{itemize}
	\item бестиповый язык;
	\item символьная обработка информации;
	\item любая программа может интерпретироваться как функция с одним или несколькими аргументами;
	\item автоматизированное динамическое распределение памяти, которая выделяется блоками;
	\item программа может быть представлена как данные, то есть программа может изменять саму себя.
\end{itemize}

Символ апостроф~--- сокращенное обозначение функции \texttt{quote}, блокирующей вычисление своего аргумента.

\section{Базис языка Lisp. Ядро языка.}

Базис языка~--- минимальный набор конструкций и структур данных, с помощью которого можно написать любую программу.

Базис Lisp образуют:
\begin{itemize}
	\item атомы;
	\item структуры;
	\item базовые функции;
	\item базовые функционалы.
\end{itemize}
