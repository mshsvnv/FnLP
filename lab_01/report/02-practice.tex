\chapter{Практические задания}

\section{Задание 1}
Представить следующие списки в виде списочных ячеек:
\begin{enumerate}
	\item '(open close halph)
	\includeimage
	{1.1}
	{f}
	{H}
	{.8\textwidth}
	{}
	\item '((open1) (close2) (halph3))
	\includeimage
	{1.2}
	{f}
	{H}
	{.9\textwidth}
	{}
	\item '((one) for all (and (me (for you))))
	\includeimage
	{1.3}
	{f}
	{H}
	{.9\textwidth}
	{}
	\item '((TOOL)(call))
	\includeimage
	{1.4}
	{f}
	{H}
	{.6\textwidth}
	{}
	\item '((TOOL1)((call2))((sell)))
	\includeimage
	{1.5}
	{f}
	{H}
	{.9\textwidth}
	{}
	\clearpage
	\item '(((TOOL)(call))((sell)))
	\includeimage
	{1.6}
	{f}
	{H}
	{.9\textwidth}
	{}
\end{enumerate}

\section{Задание 2}
Используя только функции CAR и CDR написать выражения, возвращающие:
\begin{enumerate}
	\item второй;
	\includelistingpretty
	{1.lisp}
	{lisp}
	{}
	
	\item третий;
	\includelistingpretty
	{2.lisp}
	{lisp}
	{}
	
	\item четвертый.
	\includelistingpretty
	{3.lisp}
	{lisp}
	{}
\end{enumerate}

\section{Задание 3}
Что будет в результате вычисления выражений?
\begin{enumerate}
	\item (CAADR '((blue cube) (red pyramid)));
	
	red
	\item (CDAR '((abc) (def) (ghi)))
	
	nil
	\item (CADR '((abc) (def) (ghi)))
	
	(def)
	\item (CADDR '((abc) (def) (ghi)))
	
	(ghi)
\end{enumerate}

\section{Задание 4}
Напишите результат вычисления выражений и объясните как он получен:
\includelistingpretty
{4.lisp}
{lisp}
{}

\section{Задание 5}
Написать лямбда-выражения и соответствующую функцию:
\begin{itemize}
	\item функция (f ar1 ar2 ar3 ar4), возвращающая ((ar1 ar2)(ar3 ar4));
	\includelistingpretty
	{func1.lisp}
	{lisp}
	{}
	\includeimage
	{5.1}
	{f}
	{H}
	{.9\textwidth}
	{}
	\item функция (f ar1 ar2), возвращающая ((ar1)(ar3));
	\includelistingpretty
	{func2.lisp}
	{lisp}
	{}
	\includeimage
	{5.2}
	{f}
	{H}
	{.6\textwidth}
	{}
	\clearpage
	\item функция (f arl), возвращающая (((arl)));
	\includelistingpretty
	{func3.lisp}
	{lisp}
	{}
	\includeimage
	{5.3}
	{f}
	{H}
	{.4\textwidth}
	{}
\end{itemize}