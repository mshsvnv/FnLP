\chapter{Практические задания}

\section{Задание 1}
\includeimage
{1}
{f}
{H}
{.9\textwidth}
{}

\clearpage

\section{Задание 2}
Написать функцию, вычисляющую гипотенузу прямоугольного треугольника по заданным катетам и составить диаграмму ее вычисления.

\includelistingpretty
{2.lisp}
{lisp}
{}

\includeimage
{2}
{f}
{H}
{.9\textwidth}
{}

\clearpage

\section{Задание 3}
Каковы результаты вычисления следующих выражений? (объяснить возможную ошибку и варианты ее устранения)

\begin{itemize}
	\item \texttt{(list 'a c)}
	
	Ошибка, т.к. С не самовичисляема. 
	
	Исправление: \texttt{(list 'a 'c)}. 
	
	Ответ: \texttt{(A C)}
	
	\item \texttt{(cons 'a (b c))}
	
	Ошибка, т.к. попытается вычислить \texttt{(b c)}. 
	
	Исправление: \texttt{(cons 'a '(b c))}. 
	
	Ответ: \texttt{(A B C)}
	
	\item \texttt{(cons 'a '(b c))}
	
	Ответ: \texttt{(A B C)}
	\item \texttt{(caddr (1 2 3 4 5))}
	
	Ошибка, т. к. \texttt{(1 2 3 4 5)} надо представить как список, т. е. \texttt{'(1 2 3 4 5)}.
	
	Ответ: \texttt{3}
	
	\item \texttt{(cons 'a 'b 'c)}
	
	Ошибка, т. к. \texttt{(cons)} принимает только 2 аргумента.
	
	Исправление: \texttt{(cons 'a '(b c))}
	
	Ответ: \texttt{(A B C)}
		
	\item \texttt{(list 'a (b c))}
	
	Ошибка, т. к. список \texttt{(b c)} надо представить как \texttt{'(b c)}.
	
	Исправление \texttt{(list 'a '(b c))}
	
	Ответ: \texttt{(A (B C))}
	
	\item \texttt{(list a '(b c))}
	
	Ошибка, т. к. \texttt{a} надо представить как \texttt{'a}.
	
	Исправление: \texttt{(list 'a '(b c))}
	
	Ответ: \texttt{(A (B C))}
	
	\item \texttt{(list (+ 1 '(length '(1 2 3))))}
	
	Ошибка, т. к. \texttt{'(length '(1 2 3))} не является числом.
	
	Исправление: \texttt{(list (+ 1 (length '(1 2 3))))}
	
	Ответ: \texttt{(4)}
\end{itemize}

\section{Задание 4}
Написать функцию longer\_then от двух списков-аргументов, которая возвращает Т, если первый аргумент имеет большую длину.

\includelistingpretty
{5.lisp}
{lisp}
{}

\section{Задание 5}
Каковы результаты вычисления следующих выражений:

\begin{itemize}
	\item \texttt{(cons 3 (list 5 6))}
	
	Ответ: \texttt{(3 5 6)}
	
	\item \texttt{(list 3 'from 9 'lives (- 9 3)))}
	
	Ответ: \texttt{(3 FROM 9 LIVES 6)}
	
	\item \texttt{(+ (length for 2 too))(car '(21 22 23)))}
	
	Исправление: \texttt{(+ (length '(for 2 too))(car '(21 22 23)))}
	
	Ответ: 24
	
	\item \texttt{(cdr '(cons is short for ans))}
	
	Ответ: \texttt{(IS SHORT FOR ANS)}
	
	\item \texttt{(cons 3 '(list 5 6))}
	
	Ответ: \texttt{(3 LIST 5 6)}
	
	\item \texttt{(car (list one two))}
	
	Исправление: \texttt{(car (list 'one 'two))}
	
	Ответ: ONE
	
	\item \texttt{(car (list 'one 'two))}
	
	Ответ: ONE
\end{itemize}

\section{Задание 6}

Дана функция~\ref{lst:7.lisp}. Какие результаты вычисления следующих выражений?

\includelistingpretty
{7.lisp}
{lisp}
{}

\begin{itemize}
	\item \texttt{(mystery (one two))}
	
	Исправление: \texttt{(mystery '(one two))} 
	
	Ответ: (TWO ONE)
	
	\item \texttt{(mystery (last one two))}
	
	Исправление: \texttt{(mystery '(last one two))} 
	
	Ответ: (ONE LAST)
	
	\item \texttt{(mystery free))}
	
	Исправление: \texttt{(mystery '(free))} 
	
	Ответ: (NIL FREE)
	
	\item \texttt{(mystery one 'two))}
	
	Исправление: \texttt{(mystery '(one two))} 
	
	Ответ: (TWO ONE)
\end{itemize}

\section{Задание 7}
Написать функцию, которая переводит температуру в системе Фаренгейта температуру по Цельсию (defun f-to-c (temp)...).

Формулы: c = 5/9 * (f - 32.0); f = 9/5 * c + 32.0.

Как бы назывался роман Р.Брэдбери "+451 по Фаренгейту" в системе по Цельсию?

\includelistingpretty
{8.lisp}
{lisp}
{}

\section{Задание 8}
Что получится при вычисления каждого из выражений?

\begin{itemize}
		\item \texttt{(list 'cons t NIL)}
		
		Ответ: \texttt{(CONS T NIL)}
		
		\item \texttt{(eval (eval (list 'cons t NIL)))}
		
		Ответ: ошибка
		
		\item \texttt{(eval (list 'cons t NIL))}
		
		Ответ: (T)
		
		\item \texttt{(apply \#cons ''(t Nil)}
		
		Исправление: \texttt{(apply \#'cons '(t Nil)}
		
		Ответ: \texttt{(T)}
		
		\item \texttt{(list 'eval NIL)}
		
		Ответ: \texttt{(EVAL NIL)}
		
		\item \texttt{(eval NIL)}
		
		Ответ: \texttt{NIL}
		
		\item \texttt{(eval (list 'eval NIL))}
		
		Ответ: \texttt{(NIL)}
\end{itemize}
