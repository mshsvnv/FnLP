\chapter{Практические задания}
%
%\includelistingpretty
%{1.lisp}
%{lisp}
%{}


\section{Задание 1}
Чем принципиально отличаются функции \texttt{cons}, \texttt{list},  \texttt{append}?

Каковы результаты вычисления следующих выражений?

\section{Задание 2}
Каковы результаты вычисления следующих выражений, и почему?

\section{Задание 3}
Написать два варианта функции, которая возвращает последний элемент своего списка-аргумента.

\section{Задание 4}
Написать два варианта функции, которая возвращает свой список аргумента без последнего элемента.

\section{Задание 5}
Написать функцию swap-first-last, которая переставляет в списке-аргументе первый и последний элемент.

\section{Задание 6}
Написать простой вариант игры в кости, в котором бросается две правильные кости. Если сумма выпавших очков равна 7 или 11 --- выигрыш, если выпало (1, 1) или (6, 6) --- игрок имеет право снова бросить кости, во всех остальных случаях ход переходит ко второму игроку, но запоминается сумма выпавших очков. Если второй игрок не выигрывает абсолютно, то выигрывает тот игрок, у которого больше очков. Результаты игры и значения выпавших костей выводить на экран с помощью \texttt{print}.

\section{Задание 7}
Написать функцию, которая по своему списку-аргументу \texttt{lst} определяет является ли он палиндромом.

\section{Задание 8}
Напишите \textbf{свои} необходимые функции, которые обрабатывают таблицу из 4-х точечных пар: (страна . столица), и возвращает по стране --- столицу, а по столице --- страну.

\section{Задание 9}
Написать функцию, которая умножает на заданное число-аргумент первый числовой элемент списка их заданного 3-х элементного списка-аргумента, когда
\begin{enumerate}
	\item все элементы списка --- числа;
	\item элементы списка --- любые объекты.
\end{enumerate}

