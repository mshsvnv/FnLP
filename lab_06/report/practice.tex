\chapter{Практические задания}

\section{Задание 1}
Написать хвостовую рекурсивную функцию my-reverse, которая развернет верхний уровень своего списка-аргумента lst.

\includelistingpretty
{1.lisp}
{lisp}
{}


\section{Задание 2}
Написать функцию, которая возвращает первый элемент списка-аргумента, который сам является непустым списком.

\includelistingpretty
{2.lisp}
{lisp}
{}


\section{Задание 3}
Написать функцию, которая выбирает из заданного списка только те числа, которые больше 1 и меньше 10.

\includelistingpretty
{3.lisp}
{lisp}
{}


\section{Задание 4}
Написать рекурсивную функцию, которая умножает на заданное число-аргумент все числа из заданного списка-аргумента, когда
\begin{itemize}
	\item все элементы списка~--- числа,
	\item элементы списка~--- любые объекты.
\end{itemize}

\includelistingpretty
{4.lisp}
{lisp}
{}


\section{Задание 5}
Написать функцию, select-between, которая из списка-аргумента, содержащего только числа, выбирает только те, которые расположены между двумя указанными числами~--- границами~-~аргумента и возвращает их в виде списка (упорядоченного по возрастанию).

\includelistingpretty
{5.lisp}
{lisp}
{}


\section{Задание 6}
Написать рекурсивную версию (с именем rec-add) вычисления суммы чисел заданного списка:
\begin{itemize}
	\item одноуровневого смешанного,
	\item структурированного.
\end{itemize}

\includelistingpretty
{6.lisp}
{lisp}
{}

\section{Задание 7}
Написать рекурсивную версию с именем recnth функции nth.

\includelistingpretty
{7.lisp}
{lisp}
{}

\section{Задание 8}
Написать рекурсивную функцию allodd, которая возвращает t, когда все элементы списка нечетные.

\includelistingpretty
{8.lisp}
{lisp}
{}


\section{Задание 9}
Написать рекурсивную функцию, которая возвращает первое нечетное число из списка, возможно, создавая некоторые вспомогательные функции.

\includelistingpretty
{9.lisp}
{lisp}
{}

\section{Задание 10}
Используя cons-дополняемую функцию с одним тестом завершения, написать функцию, которая получает как аргумент список чисел, а возвращает список квадратов этих чисел в том же порядке.

\includelistingpretty
{10.lisp}
{lisp}
{}