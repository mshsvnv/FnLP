\chapter{Практические задания}

\section{Задание 1}
Написать функцию, которая принимает целое число и возвращает первое четное число, не меньшее аргумента.

\includelistingpretty
{1.lisp}
{lisp}
{}

\section{Задание 2}
Написать функцию, которая принимает число и возвращает число того же знака, но с модулем на 1 больше модуля аргумента.

\includelistingpretty
{2.lisp}
{lisp}
{}

\section{Задание 3}
Написать функцию, которая принимает два числа и возвращает список из этих чисел, расположенных по возрастанию.

\includelistingpretty
{3.lisp}
{lisp}
{}

\clearpage

\section{Задание 4}
Написать функцию, которая принимает три числа и возвращает Т только тогда, когда первое число расположено между вторым и третьим.

\includelistingpretty
{4.lisp}
{lisp}
{}

\section{Задание 5}
Каков результат вычисления следующих выражений?

\includelistingpretty
{5.lisp}
{lisp}
{}

\section{Задание 6}
Написать предикат, который принимает два числа-аргумента и возвращает Т, если первое число не меньше второго.

\includelistingpretty
{6.lisp}
{lisp}
{}

\clearpage

\section{Задание 7}
Какой вариант из следующих двух вариантов предиката ошибочен и почему?

\includelistingpretty
{7.lisp}
{lisp}
{}

Ошибочным является \texttt{pred2}, так как сначала происходит проверка на положительность аргумента, а только потом является ли его аргумент числовым атомом.

\section{Задание 8}
Решить задачу 4, используя для ее решения конструкции: только IF, только COND, только AND/OR.

\includelistingpretty
{8.lisp}
{lisp}
{}

\section{Задание 9}
Переписать функцию \texttt{how-alike}, приведенную в лекции и использующую COND, используя только конструкции IF, AND/OR.

\includelistingpretty
{9.lisp}
{lisp}
{}

