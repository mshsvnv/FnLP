\chapter{Практические задания}

\section{Задание 1}
Чем принципиально отличаются функции \texttt{cons}, \texttt{list},  \texttt{append}?

\textbf{Ответ:} 
\begin{itemize}
	\item \texttt{cons}~--- базовая функция, которая объединяет значения двух своих аргументов в точечную пару.
	\item \texttt{list}~--- принимает произвольное число аргументов и возвращает список, состоящий из значений аргументов. 
	\item \texttt{append}~--- объединяет списки в один, при этом оригинальные списки остаются нетронутыми, и возвращается новый список, содержащий комбинированные элементы.
\end{itemize}

Каковы результаты вычисления следующих выражений?

\includelistingpretty
{1.lisp}
{lisp}
{}

\section{Задание 2}
Каковы результаты вычисления следующих выражений, и почему?

\includelistingpretty
{2.lisp}
{lisp}
{}

\clearpage

\textbf{Ответ:}
\begin{itemize}
	\item \texttt{reverse} работает не разрушающим образом и не изменяет исходный список, а возвращает новый список с элементами в обратном порядке. Работает только с элементами верхнего уровня; 
	\item \texttt{last} используется для извлечения последнего элемента из списка. Она принимает один аргумент~--- список и возвращает последний элемент этого списка. Работает только с элементами верхнего уровня;
\end{itemize}

\section{Задание 3}
Написать два варианта функции, которая возвращает последний элемент своего списка-аргумента.

\includelistingpretty
{3.lisp}
{lisp}
{}

\section{Задание 4}
Написать два варианта функции, которая возвращает свой список аргумента без последнего элемента.

\includelistingpretty
{4.lisp}
{lisp}
{}

\section{Задание 5}
Написать функцию swap-first-last, которая переставляет в списке-аргументе первый и последний элемент.

\includelistingpretty
{5.lisp}
{lisp}
{}

\section{Задание 6}
Написать простой вариант игры в кости, в котором бросается две правильные кости. Если сумма выпавших очков равна 7 или 11~--- выигрыш, если выпало (1, 1) или (6, 6)~--- игрок имеет право снова бросить кости, во всех остальных случаях ход переходит ко второму игроку, но запоминается сумма выпавших очков. Если второй игрок не выигрывает абсолютно, то выигрывает тот игрок, у которого больше очков. Результаты игры и значения выпавших костей выводить на экран с помощью \texttt{print}.

\includelistingpretty
{6.lisp}
{lisp}
{}

\section{Задание 7}
Написать функцию, которая по своему списку-аргументу \texttt{lst} определяет является ли он палиндромом.

\includelistingpretty
{7.lisp}
{lisp}
{}

\section{Задание 8}
Напишите \textbf{свои} необходимые функции, которые обрабатывают таблицу из 4-х точечных пар: (страна . столица), и возвращает по стране~--- столицу, а по столице~--- страну.

\includelistingpretty
{8.lisp}
{lisp}
{}

\section{Задание 9}
Написать функцию, которая умножает на заданное число-аргумент первый числовой элемент списка из заданного 3-х элементного списка-аргумента, когда
\begin{enumerate}
	\item все элементы списка~--- числа;
	\item элементы списка~--- любые объекты.
\end{enumerate}

\includelistingpretty
{9.lisp}
{lisp}
{}
