\chapter{Практические задания}

\section{Задание 1}
Напишите функцию, которая уменьшает на 10 все числа из списка-аргумента этой функции, проходя по верхнему уровню списковых ячеек.

\includelistingpretty
{1.lisp}
{lisp}
{}


\section{Задание 2}
Написать функцию, которая получает как аргумент список чисел, а возвращает список квадратов этих чисел в том же порядке.

\includelistingpretty
{2.lisp}
{lisp}
{}


\section{Задание 3}
Написать функцию, которая умножает на заданное число-аргумент все числа из заданного списка-аргумента, когда
\begin{itemize}
	\item все элементы~--- числа,
	\item элементы списка~--- любые объекты.
\end{itemize}

\includelistingpretty
{3.lisp}
{lisp}
{}

\section{Задание 4}
Написать функцию, которая по своему списку-аргументу \texttt{lst} определяет является ди он палиндромом (то есть равны ли \texttt{lst} и \texttt{(reverse lst))}, для одноуровнего смешанного списка.

\includelistingpretty
{4.lisp}
{lisp}
{}

\section{Задание 5}
Используя функционалы, написать предикат \texttt{set-equal}, который возвращает t, если два его множества-аргумента содержат одни и те же элементы, порядок которых не имеет значения.

\includelistingpretty
{5.lisp}
{lisp}
{}

\section{Задание 6}
Написать функцию, \texttt{select-between}, которая из списка-аргумента, содержащего только числа, выбирает только те, которые расположены между двумя указанными числами~--- границами~-~аргумента и возвращает из в виде списка (упорядоченного по возрастанию).

\includelistingpretty
{6.lisp}
{lisp}
{}

\section{Задание 7}
Написать функцию, вычисляющую декартово произведение двух своих списков-аргументов.

\includelistingpretty
{7.lisp}
{lisp}
{}

\section{Задание 8}
Почему так реализовано \texttt{reduce}, в чем причина?

\includelistingpretty
{8.lisp}
{lisp}
{}

Результаты в данном случае обусловлены тем, что в функции reduce в Common Lisp, если последовательность пуста, то возвращается начальное значение, указанное как аргумент \texttt{:initial-value}. 
В противном случае возвращается начальное значение первого элемента последовательности.

Поэтому, когда \texttt{reduce} применяется к пустой последовательности, начальное значение становится результатом. 

Таким образом:
\begin{itemize}
	\item \texttt{(reduce \#'+ '())} возвращает 0, так как начальное значение для сложения равно 0;
	\item \texttt{(reduce \#'* '())} возвращает 1, так как начальное значение для умножения равно 1.
\end{itemize}

\section{Задание 9}
Пусть \texttt{list-of-list} список, состоящий из списков. Написать функцию, которая вычисляет сумму длин всех элементов.

\includelistingpretty
{9.lisp}
{lisp}
{}
