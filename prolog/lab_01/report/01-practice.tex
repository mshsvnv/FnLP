\chapter{Практические задания}

\section{Задание}
Разработать свою программу~--- <<Телефонный справочник и автомобили>>. 

<<Телефонный справочник>>:
\begin{itemize}
	\item Фамилия;
	\item №тел.;
	\item Адрес(Город, Улица, №дома, №кв.);
\end{itemize}

<<Автомобили>>:
\begin{itemize}
	\item Фамилия владельца;
	\item Марка;
	\item Цвет;
	\item Стоимость;
	\item Номер.
\end{itemize}

\includelistingpretty
{main.pl}
{prolog}
{Код программы}

\section{Тестирование}
\includelistingpretty
{1.pl}
{prolog}
{Результат работы программы с одним ответом}

\includelistingpretty
{2.pl}
{prolog}
{Результат работы программы с двумя ответом}

\section{Вопросы}
\begin{enumerate}
	\item Что собой представляет программа <<Телефонный справочник>> на Prolog?
	Она представляет собой набор фактов с правилами, обеспечивающими получение заключений на основе этих фактов.
	
	\item Какова ее структура?
	\begin{itemize}
		\item \textbf{DOMAINS}~--- раздел для описания доменов;
		\item \textbf{PREDICATES}~--- раздел для описания предикатов;
		\item \textbf{CLAUSES}~--- раздел для описания предложений базы знаний;
		\item \textbf{GOAL}~--- раздел для описания цели (вопроса).
	\end{itemize}
	
	\item Как формируется результат работы программы?
	
	Система в процессе поиска подбирает ответ (<<да>>, <<нет>>) и значения, дающие ответ <<да>>. В ходе получения ответа система подбирает знания, то есть сопоставляет термы.
\end{enumerate}
